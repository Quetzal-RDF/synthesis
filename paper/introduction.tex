\section{Introduction}

SQL is arguably the de-facto query language for structured data, with an increasing number of non-relational databases such as Apache Spark \cite{Apache_Spark}, Apache Flink \cite{Apache_Flink} and Cassandra \cite{Cassandra} that either support SQL directly or some variant of SQL such as Hive \cite{Hive} or CQL \cite{CQL}. This attests to the expressive power of SQL, but the language itself is challenging for most users to use, especially for non-technical users.  As data science becomes increasingly popular, and there is an increasing need for better tooling for non-technical users to be able to manipulate and explore their data.  In fact, even seasoned data scientists would likely appreciate better support to ease the burden of generating complex SQL.

There have been many past attempts to make complex querying capabilities accessible to non-technical users.  For instance, the classic query by example approach combines examples from the user with a graphical query language to construct queries \cite{Zloof}.  Others have tried to provide natural language interfaces to databases \cite{Fei-Li}, \cite{Androutsopoulos}, \cite{Seq2SQL}, or explored the combination of keyword style queries with machine learning to construct queries \cite{Bergamaschi}.  Still others have tried to synthesize SQL using input and output examples \cite{BodikPLDI}, \cite{BodikSIGMOD}, \cite{Zhang}.  In this paper, we examine a novel approach to help users synthesize SQL expressions from input and output examples.  Our approach differs from prior work in this space in two important ways.  We focus on SQL expressions, but extend support for such expressions well beyond prior work, and we provide a mechanism to \emph{generate} input and output examples to make synthesis practical for complex expressions.  We elaborate on each difference below.

To the best of our knowledge, our approach is the first to support the synthesis of SQL expressions with a \emph{rich set of datatypes and functions} that are commonly supported by database vendors.  Prior work in synthesis has been restricted to mostly integer and boolean data types, and a highly simplified set of SQL operators on these types (e.g., comparison or boolean operators).  Yet SQL expressions are normally composed of a large number of functions on datatypes that include integers, strings, dates, and reals.  SQL functions on these types can also be composed; for instance arbitrarily complex arithmetic functions can be composed in SQL, but these have not been addressed in past work.  Our observation is that we can actually extend synthesis to richer datatypes by relying heavily on solver technologies; specifically we leverage the Z3 SMT solver \cite{Z3} which can solve for constraints involving strings and reals (along with the traditional boolean or boolean types supported by SAT solvers).  Another key component of our solution is Rosette \cite{Rosette}, Rosette is a solver-aided programming language that allows us to translate SQL functions into code.  Rosette compiles these functions to logical constraints solved with Z3, and thus enables support for both complex custom functions needed for SQL, along with the rich datatype support that Z3 provides. We show that by building on these two sets of technologies, one can support a large set of SQL functions on rich datatypes. 

Second, we describe a mechanism to use the same machinery we use for synthesis of SQL to \emph{generate} input and output examples if the user can provide the system with snippets from the SQL expression.  We use the example in Figure \ref{fig:1} to illustrate this with an example.  Figure \ref{fig:1} shows a CASE expression.  For most non-technical users, this is a difficult expression to express syntactically, but users have no trouble expressing the snippets of the query such as the fact that the Customer Column needs to be `Regular' or  `Standard' to be interesting.  Examples of snippets a user might be able to express easily are provided in Figure \ref{fig:1}.  Our work can use any of the snippets the user can provide to generate input and output examples for users to correct rather than having the user generate the examples themselves.  This not only reduces the burden on the user to provide enough examples but additionally allows the generation of the right set of examples for the problem.  In fact, our observation is that left to their own devices, many users (including the authors) often generate underspecified examples, which result in the synthesis of expressions that are valid for the examples, but clearly not what the user intended.  Table \ref{table:1} shows an example of what a user might enter as plausible input and output examples for the synthesis of the expression in Figure \ref{fig:1}.  However, the examples are not sufficient to constrain the problem because the last row can be interpreted as satisfying either branch of the CASE statement.  Being able to generate input and output examples from snippets allows users to focus on just correcting the data and lets the synthesis process generate the fabric to stitch the snippets together. 

Snippets also allows synthesis to scale better, given that the problem of searching over all possible combinations of functions, even to a limited depth is not very practical.  Snippets generalize ideas in earlier work \cite{BodikPLDI} which asks users to specify the constant terms in the query such as `Standard' in the example shown in Figure \ref{fig:1} to help narrow the search space.  We provide a mechanism to incorporate these snippets into the search so that when users are finished with their corrections of the examples provided by the system, the system uses snippets along with the examples to synthesize the SQL expression.

We evaluate our ideas against a large benchmark of SQL expressions mined from XX SQL queries gathered from SQL files in Google's public GitHub \cite{GitHub_repo} dataset.  INSERT SOME RESULTS HERE from the benchmark.  We make this benchmark and our code for synthesis and input output generation available at GitHub \cite{Quetzal}.

To summarize, our contributions are as follows:
\begin{enumerate}
\item We describe a novel approach to the problem of SQL synthesis which allows us to extend synthesis for rich datatypes and complex SQL functions that are common in SQL queries.
\item We describe an approach to automatically generate the input output examples needed for synthesis if users can specify any snippets in the expression.  
\item We illustrate how those snippets can be incorporated along with the corrected input output examples into synthesis to make it more scalable and to have the synthesis process quickly converge on the solution the user had in mind.
\item We provide a comprehensive real life benchmark of SQL expressions by mining GitHub, and use these expressions to evaluate our own approach to input-output example creation, and the problem of synthesizing SQL expressions for rich datatypes and functions.
\end{enumerate}

The paper is organized as follows.  Section \ref{prelims} outlines the types of SQL functions and datatypes that are supported in our current synthesis approach, and what in general limits support for functions over these datatypes using solver technologies.  Section \ref{section:2} also gives a background of how Rosette along with SMT solvers can be harnessed to solve this problem.  Section \ref{section:3} covers the algorithms for input output generation.  Section \ref{section:4} describes the algorithm used in search, including the use of existing snippets.  Section \ref{section:5} describes the evaluation, as well as details about how the benchmark was generated.  Section \ref{section:6} summarizes the related work, and Section \ref{section:7} presents the conclusions and future work.

\begin{figure}
\caption{An example SQL expression}
\label{fig:1}
\begin{lstlisting}[
           language=SQL,
           showspaces=false,
           basicstyle=\ttfamily,
           numbers=left,
           numberstyle=\footnotesize,
           commentstyle=\color{gray}
  ]
  CASE WHEN Customer='Regular'
     THEN price * product_qty
  WHEN Customer='Bulk' 
     THEN bulk_price * case_qty
  ELSE 0
\end{lstlisting}

Snippets: 
$Customer=$`$Regular$', $Customer=$`$Bulk$'
$bulk\_price * case\_qty$, $price * product\_qty$
\end{figure}

\begin{table}[h!]
\centering
\begin{tabular}{ |l|r|r|r|r|r| }
\hline
Customer & bulk price & case qty & price & product qty & out \\
Bulk & 3 & 4 & 0 & 0 & 12 \\
Regular & 0 & 0 & 2 & 3 & 6 \\
Foo & 0 & 0 & 0 & 0 & 0 \\
\hline
\end{tabular}
\caption{Table of user specified inputs and output}
\label{table:1}
\end{table}


